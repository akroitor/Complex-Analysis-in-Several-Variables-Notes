% !TEX root = ../CO739.tex
\newpage
\section{Review of Complex Analysis in one variable.}

Before we start with several variables, we give a short review of single variable complex analysis.

\subsection{Holomorphic Functions}

\begin{definition}
    We call $\om \subset \C$ a domain if $\om$ is open and connected.
\end{definition}

\begin{definition}
    Let $u \in C^1 ( \om)$, with $z = x + i y$, $x,y \in \R, z \in \R$. Then $u $ is a complex function of $x ,y$. We define the exterior derivative of $u$ as
    \begin{align*}
        \dif u &= \frac{\partial u }{\partial x } \dif x + \frac{\partial u}{\partial y} \dif y \qquad \text{(this is a 1-form)}, \\
        &=: \frac{\partial u}{\partial z} \dif z + \frac{\partial u}{ \partial \ov{z} } \dif \ov{z},
    \end{align*}
    where
    \begin{align*}
        \frac{\partial u}{\partial z} &= \frac{1}{2} \br{\frac{\partial u}{\partial x} - i \frac{\partial u}{\partial y}},\\
        \frac{\partial u}{\partial \ov{z}} &= \frac{1}{2} \br{\frac{\partial u}{\partial x} + i \frac{\partial u}{\partial y}},\\
        \dif z &= \dif x + i \dif y,\\
        \dif \ov{z} &= \dif x - i \dif y.
    \end{align*}
\end{definition}

\begin{definition}
    $u \in C^1 (\om )$ is holomorphic if $\frac{\partial u}{\partial \ov{z}} = 0 $ on $\om$.
\end{definition}

\begin{note}
    Playing with the definition of $\frac{\partial u}{\partial \ov{z}}$ gives the Cauchy-Riemann equations as an equivalent definition:
    \begin{align*}
        \frac{\partial \Re u}{\partial x } &= \frac{\partial \Im u}{\partial y}\\
        \frac{\partial \Im u}{\partial x } &= -\frac{\partial \Re u}{\partial y}.
    \end{align*}
\end{note}

\begin{note}
    Differentiability of $f$ is a property at a point, but holomorphicity is a property of $f$ on an open set (not just a point).
\end{note}

\begin{definition}
    Define
    \begin{align*}
        \O (\om) = \set{ \text{holomorphic functions on $\om$} } \qquad \text{(also written $A(\om)$)}.
    \end{align*}
\end{definition}

\begin{example}
    Holomorphic functions include:
    \begin{itemize}
        \item $z^n$
        \item $e^z$
        \item Products, sums, compositions of holomorphic functions are holomorphic. Notably polynomials in $z$ are holomorphic, and rational functions $\frac{p(z)}{q(z)}$ are holomorphic away from the zeroes of $q$ (called poles of $\frac{p(z)}{q(z)}$).
    \end{itemize}
\end{example}
\begin{nexample}
    A function which is not holomorphic is $f(z) = z \ov{z} = \abs{z}^2$. Morally this is since $\ov{z}$ is in the formula for $f$.
\end{nexample}

\begin{recall}
    Recall that if $u :\R^2_{(x,y)} \to \R^2_{(x',y')}$ with $u = (u_1,u_2)$ then the Jacobian of $u$ is
    \begin{align*}
        \jac(u) = \det \underbrace{ \left| \begin{matrix}
            u_{1x} & u_{1y}\\
            u_{2x} & u_{2y}
        \end{matrix} \right| }_{Du}.
    \end{align*}
    If $\jac u (x_0,y_0) \neq 0$ then $u $ is locally invertible and $u^{-1}$ is differentiable with
    \begin{align*}
        \br{Du^{-1} } (u(x_0,y_0)) = \left[ Du(x_0,y_0) \right]^{-1}.
    \end{align*}
\end{recall}

\begin{exercise}
    A holomorphic function $f : \C \to \C$ can be thought of as a map $\tilde{f} : \R^2 \to \R^2$. If
    \begin{align*}
        \frac{\partial f}{\partial z} (z_0) =: f'(z_0) \neq 0
    \end{align*}
    then prove that $f$ is invertible and that
    \begin{align*}
        \jac \tilde{f} (z_0) = \abs{f'(z_0)}^2.
    \end{align*}
\end{exercise}

\subsection{Cauchy Integral Formula}
Probably the most important result in complex analysis in one dimension is the so-called Cauchy integral formula.

\begin{theorem}
    Let $u \in C^1(\om)$, $\om $ bounded, $\bdy \om$ piecewise smooth. Then $\forall z \in \om$
    \begin{align*}
        u(z) = \frac{1}{2 \pi i} \int_{\bdy \om} \frac{u (\zeta) }{\zeta - z } \dif \zeta - \frac{1}{\pi} \iint_{\om} \frac{\partial u / \partial \ov{z}}{\zeta - z } \dif \zeta \dif \ov{\zeta}.
    \end{align*}
    Note that formally $\dif \zeta \dif \ov{\zeta}$ should have a wedge product to be $\dif \zeta \wedge \dif \ov{\zeta}$.
\end{theorem}

\begin{proof}
    This is an application of Green's formula (which itself is an application of Stokes' Theorem). More or less you realize that $z $ is a pole of $\frac{\partial u / \partial \ov{z}}{\zeta - z }$, then define $\om_\epsilon = \om - \ball{z}{\epsilon}$ and apply Green's Formula to $\om_\epsilon$.
\end{proof}

\begin{remark}
    If $u$ is holomorphic on $\om$, then $\partial u / \partial \ov{z} = 0$. Then
    \begin{align*}
        u(z) = \frac{1}{2 \pi i} \int_{\bdy \om} \frac{u (\zeta) }{\zeta - z } \dif \zeta.
    \end{align*}
    This is called the Cauchy Integral Formula. Note that the integrand depends holomorphically on $z$. Morally this says that values of $u$ on $\bdy \om$ completely determines $u $ on $\om$.
\end{remark}

\begin{remark}
    $\frac{1}{\zeta - z}$ is the Cauchy kernel. This is some sense a ``holomorphic reproducing kernel", since integrating a holomorphic function against it will give another holomorphic function.
\end{remark}

\begin{corollary}
    Let $u \in \O(\om)$. Then $u \in C^\infty(\om)$.
\end{corollary}

\begin{proof}
    We can pull the derivative under the integral. In particular we have that $u'(z) \in \O(\om)$.
\end{proof}

\begin{corollary}
    Let $u \in \O(\om) , z_0 \in \om$. Let $r>0$ be small such that $\ball{z_0}{r} \subset \om$. Then in $\ball{z_0}{r}$ we have a power representation for $u(z)$:
    \begin{align*}
        u(z) &= \sum_{n \geq 0} c_n (z-z_0)^n,\\
        c_n &= \frac{1}{2 \pi i } \int_{\bdy \om} \frac{u(\zeta) }{(\zeta - z)^{n+1}} \dif \zeta.
    \end{align*}
    This means that $u \in \O(\om) \implies u$ is real analytic.
\end{corollary}

\begin{proof}
    We write
    \begin{align*}
        \frac{1}{\zeta - z} = \frac{1}{(\zeta - z-0) \br{ 1 - \frac{z-z_0}{\zeta - z_0}}} = \sum_{n \geq 0} \frac{(z-z_0)^n}{(\zeta  - z_0)^{n+1}}.
    \end{align*}
    Then we put this back into the cauchy Integral Formula. This converges since $\ball{z_0}{r} \subset \om$.
\end{proof}

Note that the converse of this also holds.
\begin{corollary}
    If $\sum_{n \geq 0} a_n (z-z_0)^n$ converges on $\ball{z_0}{r}, r \geq 0$, then the the sum is holomorphic.
\end{corollary}

In order to prove this we need the following results.

\begin{theorem}
    If $f \in \O(\om) \cap C(\ov{\om})$ then we have that
    \begin{align*}
        \int_{\bdy \om} f(z) \dif z = 0.
    \end{align*}
    ie $f(z) \dif z$ is a closed form (by Stokes).
\end{theorem}

\begin{theorem}
    Let $f \in C(\om)$ and $\int_{\bdy R} f(z) \dif z = 0$ for all rectangles $R$ in $\om$. Then $f \in \O(\om)$. 
\end{theorem}

\begin{note}
    If $F(z) := \int_{z_0}^z f(\zeta) \dif \zeta$, this shows that that $F \in \O (\om) \implies f \in \O(\om)$.
\end{note}

\begin{corollary}
    If $\{ u_n \} $ are holomorphic functions on $\om$ which converge uniformly, that is to say that $\norm{u - u_n}_k \to 0$ as $n \to \infty$, on any compact $K \subset \om$ to a function $u$, then $u \in \O(\om)$.
\end{corollary}

\begin{proof}
    Partial sums (polynomials) converge to the sum uniformly on compact $\ball{z_0}{r}$.
\end{proof}

\begin{exercise}
Let $u(z) \not\equiv 0$ be holomorphic near $z_0$, and let $u(z_0) = 0$. Show that
\begin{align*}
    u(z) = (z-z_0)^k g(z),
\end{align*}
for some $k < \infty$, where $g(z_0) \neq 0$.
\end{exercise}

\begin{proof}
    Use the fact that holomorphicity is equivalent to having a power series representation.
\end{proof}

% suspicious
\begin{theorem}
    Let $\om \subset \C$ be bounded, $f \in \O(\om) \cap C(\om)$. Then $\max \abs{f} $ is attained on $\bdy \om$.
\end{theorem}

\begin{quizques}
    Let $f \in \O (\C) $ be an entire function, and let $K \Subset \C$ be compact. Is it trye that $f$ can be uniformly approximated on $K$ by holomorphic polynomials?
\end{quizques}

\begin{quizans}
    True. Let $R > 0$ such that $\ball{0}{R} \supset K$, then take partial sums of the power series representation of $f$ on $\ball{0}{R}$. This gives a uniform approximation on $K$.
\end{quizans}

\subsection{Runge Theorem and Convexity}
\begin{theorem}
    Let $\om \subset \C$ be bounded and $K \Subset \om$ be compact. TFAE:
    \begin{enumerate}[a)]
        \item Every holomorphic function in a neighbourhood of $K$ can be approximated uniformly on $K$ by functions in $\O (\om)$.
        \item $\om - K$ has no components relatively compact in $\om$.
        \item $\forall z \in \om - K$, there exists $f \in \O(\om)$ such that
        \begin{align*}
            \abs{f(z)} > \sup_{w \in K} \abs{f(w)}.
        \end{align*}
    \end{enumerate}
\end{theorem}

\begin{example}
    Some drawings that I have not copied.
\end{example}

\begin{example}
    Let $\om = \C$, $K = \set{\abs{z} = 1}$. Then $f(z) = \frac{1}{z}$ is holomorphic in a neighbourhood of $K$, but $f$ cannot be approximated by entire functions, as $f\mid_K = \ov{z} $. Note that $\C - K $ has $2$ connected components, one of which ($\set{\abs{z} < 1}$) is relatively compact, so condition $b)$ fails.
\end{example}

% suspicious
\begin{proof} \phantom{.}
    \begin{itemize}
        \item[$\boxed{c) \implies b)}$]: Suppose that $b)$ is false, so $\exists A$ connected component of $\om - K$ such that $\ov{A} \Subset \om$. Then $\bdy A \subset K$, and by the maximum principle for holomorphic functions $\forall f \in \Lambda(\O)$
        \begin{align*}
            \sup_{\ov{A}} \abs{f} = \sup_{\bdy A} \abs{f} \leq \sup_{K} \abs{f},
        \end{align*}
        but this contradicts $c)$ by taking $z \in A$.
        \item[$\boxed{a) \implies b)}$] We proceed by contradiction. Suppose there exists $A \subset (\om - K)$ is relatively compact. Now let $f \in \O(K)$, and choose $\set{ f_n } \subset \O(\om)$ such that $f_n \to f$ uniformly on $K$. By the maximum principle we have that
        \begin{align*}
            \sup_{\ov{A}} \abs{f_n - f_m} = \sup_{\bdy A} \abs{f_n - f_m} \leq \sup_{K} \abs{f_n - f_m} \to 0.
        \end{align*}
        Thus $\set{f_n}$ converges uniformly on $\ov{A}$ to a holomorphic function $F \in C(\ov{A})$.

        Then pick $f(z) = \frac{1}{z-z_0}$ for some $z_0 \in A$. Then $(z-z_0) F(z) = 1$ on $\bdy A$, and so $(z-z_0) F(z) = 1$ on $A$. This implies that $F$ has a pole at $z_0$, which is a contradiction.

        \item[$\boxed{b) \implies a)}$] To prove this we need two propositions.
        \begin{proposition}
            Let $K \subset \C$ be compact. Then any $f \in \O(K)$ can be uniformly approximated on $K $ by rational functions with poles off $K$.
        \end{proposition}
        \begin{proof}
            Let $D \supset K$ be a domain with $\bdy D $ smooth such that $f \in \O(\ov{D})$. By the Cauchy integral formula
            \begin{align*}
                f(z) = \frac{1}{2 \pi i} \int_{\bdy D} \frac{f(\zeta) }{\zeta - z} \dif \zeta \qquad \forall z \in D.
            \end{align*}
            We can chop $\bdy D$ into a union of small arcs $\gamma_j$ such that $\gamma_j \subset \ball{c_j}{r_j}$ where $c_j \in \bdy D$ and $\ball{c_j}{r_j} \cap K = \emptyset$. Then $f(z) = \sum_j f_j(z) $ where
            \begin{align*}
                f_j(z) = \frac{1}{2 \pi i} \int_{\gamma_j} \frac{f(\zeta) }{\zeta - z} \dif \zeta.
            \end{align*}
            Note that $f_j$ is analytic off $\gamma_j$, as we can differentiate by $\ov{z}$ (there is no dependence on $\ov{z}$), and it vanishes at $\infty$.

            It follows that $f_j$ has a Laurent series expansion in descending powers of $(z-c_j)$ that converge uniformly on $\C - \ball{c_j}{r_j}$, especially on $K$. Thus $f_j(z)$ can be approximated on $K$ by polynomials in $\frac{1}{z-c_j}$. Adding these polynomials gives the required approximation.
        \end{proof}

        \begin{proposition}
            Let $K \subset \C$, $U \subset \ov{\C} - K$ be open and connected. Let $z_0 \in U$. Then any rational function with poles in $U$ can be approximated uniformly on $K$ by rational functions with a pole only at $z_0$.
        \end{proposition}

        \begin{proof}
            Let
            \begin{align*}
                V = \set{ \zeta \in U \mid \tfrac{1}{z - \zeta } \text{ can be approximated on $K$ by rational functions with poles at $z_0$}}.
            \end{align*}
            Then $V \subset U$ and $z_0 \in V$. If we show that $V$ is open and closed in $U$ we have that $V = U$. 
            
            Closedness is clear -- let $U \ni \zeta = \lim_j \zeta_j$ for $\zeta_j \in V$. Then $\frac{1}{1-\zeta_j} \to \frac{1}{1-\zeta}$ uniformly on $K$. 
            
            Openness is less immediate -- we need that if $\zeta_0 \in V$ and $\zeta$ is sufficiently close to $z_0$ then $\frac{1}{z-\zeta}$ is approximable on $K$ by polynomials in $\frac{1}{z-\zeta_0}$. We split into cases.
            \begin{itemize}
                \item[Case 1:] $\zeta_0 = \infty$. If $\abs{\zeta} \gg 1$ then $\frac{1}{z-\zeta} $ is uniformly approximable on $K$ by polynomials in $z$, ie rational functions with poles at $\infty$.

                Indeed expand
                \begin{align*}
                    \frac{1}{z-\zeta} = \frac{-1}{\zeta} \frac{1}{1-\frac{z}{\zeta}} = \frac{-1}{\zeta} \sum_{k \geq 0} \frac{z^k}{\zeta^k}.
                \end{align*}
                If $\abs{\zeta} \gg 1$ then $|\frac{z}{\zeta}| < \frac{1}{2}$ on $K$. By the Weierstrass M-test, the series converges uniformly on $K$, and so $\zeta \in V$.
                \item[Case 2:] If $\zeta_0$ is finite then we can change coordinates on $\ov{\C} = \C \mathbb{P}^1$ such that $z_0 \to \infty$, and then repeat the argument as in case 1. 
            \end{itemize}
        \end{proof}
        To finish the proof we let $f \in \O(K)$, approximate by rational functions, then use the moving poles proposition to move poles of approximating sequence away from $\om$.
    \end{itemize}

    For the remaining direction $b) \implies c)$ consult Hormander section 1.3.
\end{proof}

\begin{definition}
    Let $K \Subset \om \subset \C$. Define the holomorphically convex hull of $K $ in $\om$ to be
    \begin{align*}
        \hat{K}_{\om} := \set{ z \in \om \mid \abs{f(z)} \leq \sup_{K} \abs{f} , \forall f \in \O (\om)}.
    \end{align*}
\end{definition}

This is quite a strange definition at first! It is easier to conceptualize with a few remarks on it.

\begin{remark}
    If $z_0 \in \C - \om$ then $f(z) = \frac{1}{z-z_0} \in \O(\om)$, so $\operatorname{dist}(K , \bdy \om)  = \operatorname{dist}(\hat{K}_\om , \bdy \om)$.
\end{remark}
\begin{remark}
    Recall that $K \subset \R^2$ is convex if $\R^2 - K$ is the union of open halfspaces. If $K \Subset \R^2$, the convex hull of $K$ is the smallest convex compact set containing $K$.
\end{remark}
\begin{remark}
    $\hat{K}_{\om}  $ is contained in the convex hull of $K$. Why is this so? This is since 
    \begin{align*}
        \abs{e^{z}} = e^{\Re(z)},
    \end{align*}
    which gives a function, where to the left of the line $\Re(z) = 0$ the modulus is smaller than on the line, while to the right of the line the modulus is larger. Along with multiplication by constants $\alpha \in \C$ and shifts by a constant in $\C$ this gives all halfspaces.
\end{remark}

\begin{theorem}
    Let $K \Subset \om$. Then
    \begin{align*}
        \hat{K}_\om = K \cup \set{\text{relatively compact components of } \om - K}.
    \end{align*}
\end{theorem}

\begin{proof}
    Consult Hormander section 1.3. Morally use the implication $b) \implies c)$ in Runge's theorem.
\end{proof}

\begin{corollary}
    Any domain $\om \subset \C$ can be exhausted by compact sets $K_j$ (ie $K_1 \Subset K_2 \Subset K_3 \Subset \cdots$ such that $\bigcup_{j \geq 1} K_j = \om$) such that $K_j = \widehat{K_j}_\om$ (ie $K_j$ is holomorphically convex).
\end{corollary}

\begin{definition}
    A function $f$ is called meromorphic on $\om \subset \C$ if there exists a discrete set $P \subset \om$ such that $f \in \O (\om - P)$ and $\forall p \in P$, $p$ is a pole for $f$ (ie near $p$ $f$ can be written $f = \frac{h}{g}$ where $h,g$ are holomorphic near $p$).
\end{definition}

\begin{remark}
    $f$ meromorphic on $\om$ can be veiwed as a holomorphic map
    \begin{align*}
        f: \om \to \C \mathbb{P}^1 \cong \C \cup \set{\infty}.
    \end{align*}
\end{remark}

\begin{theorem}
    Let $\set{z_j}_{j=1}^\infty \subset \om$ be a discrete sequence of points, and let $f_j$ be meromorphic functions near $z_j$. Then there exists a meromorphic function $f$ in $\om$ such that $f$ is holomorphic on $\om - \bigcup_{j=1}^\infty \set{z_j}$ and $f - f_j$ is holomorphic near $z_j$ for all $j$.
\end{theorem}


\begin{proof}
    We cannot naively sum the $f_j$'s (the sum may not converge) though if there are finitely many points $z_j$ this works. Instead we use Runge's theorem to find good approximations, in the sense that if we subtract them from the sum convergence is guaranteed, while maintaining holomorphicity of $f-f_j$ near $z_j$.
\end{proof}

\begin{theorem}
    Let $\set{z_j}_{j=1}^\infty$ be a discrete subset of $\om \subset \C$, and let $n_j \in \Z$. Then there exists and meromorphic function $f $ on $\om$ such that $f \in \O(\om - \bigcup_{j} \set{z_j}), f \not\equiv 0$ on $\om$. Moreoever $f(z) \cdot (z-z_j)^{-n_j}$ is holomorphic and $\neq 0$ near $z_j$ for all $0$. That is to say that $f$ has prescribed $0$'s and poles in $\om$.
\end{theorem}

\begin{note}
    If there are a finite number of $z_j$ we can let
    \begin{align*}
        f(z) = \prod_{j < \infty} (z-z_j)^{n_j}.
    \end{align*}
    If there are infinitely many it is less straightforward.
\end{note}

\begin{remark}
    Mittag-Leffler prescribes poles and principal part of the Laurent expansions, while Weierstrass prescribes $0$'s and poles, along with degrees, but not the principal part.
\end{remark}

\begin{corollary}
    Any meromorphic function $f$ on $\om$ can be written as $f = \frac{h}{g}$ where $h,g \in \O(\om)$.
\end{corollary}

\begin{proof}
    Apply Weierstrass's theorem to poles and $0$'s of $f$ separately to get respectively $h,g$.
\end{proof}

\begin{corollary}
    Given any domain $\om \subset \C$ there exists $f \in \O (\om)$ that can't be extended beyond $\om$, evan as a meromorphic function.
\end{corollary}

\begin{proof}
    Let $\set{z_j} $ be such that $\ov{\set{z_j}}$ is dense subset of $\bdy \om$. Then use Weierstrass's theorem on these to get a function with non-separable poles.
\end{proof}

\subsection{Subharmonic Functions}

\begin{definition}
    $h : \R^2 \to \R^2$ is harmonic if $h \in C^2 (\R^2)$ and 
    \begin{align*}
        -\Delta h : = -\br{ \frac{\partial^2 h }{\partial x^2} +\frac{\partial^2 h }{\partial y^2}} = 0 \qquad \text{ on } \R^2.
    \end{align*}
\end{definition}

\begin{example}
    For $f \in \O(\om)$, $\Re f$ and $\Im f$ are harmonic functions on $ \om$. Thus if $h$ is harmonic then $h$ is $C^\infty$ (ie real-analytic).
\end{example}

\begin{theorem}
    Let $h$ be harmonic. Then
    \begin{align*}
        h(z) = \frac{1}{2 \pi} \int_0^{2 \pi} h(z+re^{i \theta}) \dif \theta.
    \end{align*}
\end{theorem}

\begin{definition}
    Let $\om \subset \C$. We say that $u : \om \to \R$ is upper semi-continuous (USC) if $\set{z \in \om \mid u(z) < s} $ is open in $\om$ for any $s \in \R$.

    Equivalently $u$ is USC at $z_0 \in \om$ if $\forall \epsilon>0$ there exists $\delta > 0 $ such that
    \begin{align*}
        \abs{ z - z_0} < \delta \implies u(z) - u(z_0) < \epsilon.
    \end{align*}
\end{definition}

\begin{note}
    If we had $\abs{u(z) - u(z_0)}$ instead of $u(z) - u(z_0)$ this would be definition of continuity. Swapping $< $ to $>$ would give the definition for lower semi-continuous.
\end{note}

\begin{example}
    Some pictures.
\end{example}

\begin{definition}
    Let $\om \subset \C$. A function $u : \om \to [-\infty, \infty)$ is called subharmonic if
    \begin{enumerate}
        \item $u$ is USC on $\om$,
        \item $\forall K \Subset \om, \forall h \in C(K)$ harmonic on $\interior{K} $ and $h \geq u$ on $\bdy K$, we have $u \leq h$ on $K$.
    \end{enumerate}
\end{definition}



\begin{note}
    In the definition above it suffices to take $K = \ov{\ball{a}{r}} \subset \om $ for some $a \in \om, r > 0$.
\end{note}

\begin{note}
    Depending on the author, $u \equiv - \infty$ is subharmonic or not. If you do consider it to be subharmonic, you need to treat it separately most times as an edge case.
\end{note}

\begin{example}
    For $f \in \O(\om)$ then $\log \abs{f}$ and $\abs{f} $ are subharmonic.
\end{example}

\begin{theorem}
    Let $SH(\om)$ be the space of subharmonic functions on $\om$.
    \begin{enumerate}
        \item For $u \in SH(\om)$ and $c>0$, $cu \in SH(\om)$.
        \item For $\set{u_\alpha}_{\alpha \in A}$ are subharmonic, let $u = \sup_{\alpha} u_\alpha$ (pointwise). If $u$ is USC and $u < \infty$ then $u$ is subharmonic.
        \item If $\set{u_j}$ is a decreasing sequence of subharmonic functions, then $u = \lim_j u_j$ is subharmonic.
    \end{enumerate}
\end{theorem}

\begin{proof}
    \phantom{.}
    \begin{enumerate}
        \item Left as an exercise.
        \item Left as an exercise.
        \item Let $s \in \R$ and consider
        \begin{align*}
            \set{z \in \om \mid u(z) < s} = \bigcup_{j} \underbrace{\set{z \in \om \mid u_j(z) < s}}_{open}.
        \end{align*}
        Then since a union of open sets is open, it follows that $u$ is USC. Now let $h$ be harmonic on some $K \subset \om$ and suppose that $u \leq h$ on $\bdy K$. Let $\epsilon > 0$. Then $\set{z \in \bdy K \mid u_j \geq h + \epsilon}$ is compact and decreasing in $j$, and so it follows that the limit is $\emptyset$. Thus $\exists j_0 $ such that $u_j \leq h+\epsilon$ for all $j \geq j_0$. Since $\epsilon$ is arbitrary this implies that $u \leq h$ in $K$ as desired.
    \end{enumerate}
\end{proof}

\begin{theorem}
    Suppose $u$ is subharmonic on $\om \subset \C$, and that $\max_{\om} u = u(z_0)$ for some $z_0 \in \om$. Then $u$ is identically a constant.
\end{theorem}

\begin{proof}
    Suppose that $u \not\equiv \text{cst}$. Then $\exists \zeta_1 \in \ov{\ball{z_0}{r}}$ with $u(\zeta_1) < u(z_0)$ where $r = \abs{z_0 - \zeta_1}$.

    Since $u $ is USC $\exists$ an arc $\gamma \ni \zeta_1$ such that
    \begin{align*}
        u(\zeta) < u(z_0) - \epsilon \qquad \text{ on $\gamma$ for some $\epsilon >0$}. 
    \end{align*}

    Take $\gamma' \Subset \gamma$ and construct a function $h \in C(\bdy \ball{z_0}{r})$ such that
    \begin{align*}
        h &= u(z_0) - \epsilon \quad \text{ on $\gamma'$},\\
        h &= u(z_0) \! \! \quad\qquad \text{ on $\bdy \ball{z_0}{r} - \gamma$}.
    \end{align*}
    Note that we can ensure that $h$ varies linearly with respect to angle on $\gamma - \gamma'$. Now let $\tilde{h}$ be a harmonic extension of $h$ to $\ball{z_0}{r}$. Note that $\tilde{h}$ comes from the Poisson integral of $h$, which gives another example of a kernel, in this case a ``harmonic" one.

    On $\bdy \ball{z_0}{r}$ we have $u \leq h$ by construction, and
    \begin{align*}
        u(z_0) \leq \tilde{h}(z_0) = \frac{1}{2 \pi} \int_{\bdy \ball{z_0}{r}} h(\zeta) \dif \zeta < u(z_0),
    \end{align*}
    where the last inequality comes from the fact that the LHS is the average of $h$ on $\bdy \ball{z_0}{r}$. This yields a contradiction, thus $u$ is identically constant.
\end{proof}

\begin{theorem}
    Let $u $ be USC on $\om$. Then
    \begin{align*}
        u \text{ is subharmonic } \iff u(z) \leq \frac{1}{2 \pi} \int_0^{2 \pi} u(z+ re^{i \theta} ) \dif \theta \quad \forall z \in \om, r>0 \text{ st. } \ball{z_0}{r} \subset \om.
    \end{align*}
\end{theorem}
\begin{proof}
    Approximate $u$ with harmonic functions on the boundary and apply the MVT for harmonic functions.
\end{proof}

\begin{corollary}
    If $f \in \O (\om)$ then $u = \ln \abs{f}$ and $u = \abs{f}$ are subharmonic.
\end{corollary}

\begin{proof}
    Apply MVT and max modulus bound.
\end{proof}

\begin{remark}
    \phantom{.}
    \begin{itemize}
        \item Let $u$ be subharmonic. Then $\set{z \mid u(z) = - \infty}$ is called the polar set of $u$. It is known that this set has Lebesgue measure $0$ (or is all of $\C$ in the edge case that $u \equiv - \infty$).
        \item If $u \in C^2(\om) $ then
        \begin{align*}
            u \text{ subharmonic } \iff -\Delta u \leq 0.
        \end{align*}
        \item Subharmonic functions can be defined on $\R^n$ for $n \geq 1$.
    \end{itemize}
\end{remark}











