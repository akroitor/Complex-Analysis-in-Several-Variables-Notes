% !TEX root = ../CO739.tex
\newpage
\section{Introduction \label{sec:intro}}

This is a course on several complex variables. Prerequisites include complex analysis in a single variable and some linear algebra. Knowledge of differential forms would be useful as well.

This is a course offered through the Fields Institute, taught by Rasul Shafikov, a professor at the University of Western Ontario.

This course will be following three books broadly. They are:
\begin{itemize}
    \item Cox-Little-O'Shea
    \item Ener-Herzog
    \item Miller-Sturmfels
\end{itemize}


Please note that these notes may have errors in them, either from my taking them down badly, or from Rasul miswriting something. Please also note the lack of pictures. This will be remedied at some point.

% Note that these notes are meant to provide brief results and not maximal detail.
% There are many famous theorems in Mathematics. One of the most famous theorems is Fermat's Last Theorem.

% \begin{thm}[Fermat's Last Theorem] \label{thm:fermat}
% If $n>2$, there are no integers $a,b,c$ with $abc \neq 0$ such that $a^n + b^n= c^n$.
% \end{thm}


% % Traditional Mathematics
% \subsection{Purpose}

% Theorem~\ref{thm:fermat} is one of the most famous theorems in Mathematics. But most undergraduate students do not learn Fermat's Last Theorem. Instead, many students learn formulas such as:
% 	\[
% 	\oint_C \mathbf{F} \cdot d\mathbf{r} = \iint_S \nabla \times \mathbf{F} \cdot dS
% 	\]
% But Mathematics starts far more simple than that. The first topic in Mathematics that one typically sees is Arithmetic. For instance, students typically will learn ``FOIL.'' 
% 	\begin{equation} \label{eq:foil}
% 	(x + y)^2= x^2 + 2xy + y^2
% 	\end{equation}
% However, \eqref{eq:foil} tends to be a stumbling block for students. Many students will instead claim, incorrectly, that $(x+y)^2= x^2 + y^2$. The purpose of this course will be to prove Dirichlet's Unit Theorem, which states:

% \begin{restatable*}[Dirichlet's Unit Theorem]{thm}{unit} \label{thm:unit}
% Let $K$ be a number field of degree $n$ with $r$ real embeddings and $s$ conjugate pairs of complex embeddings. Then the abelian group $\O_K^\times$ is a finitely generated abelian group with rank $r+s-1$ and $\O_K^\times \cong \mu_K \times \Z^{r+s-1}$, where $\mu_K$ are the roots of unity in $\O_K$. 
% \end{restatable*}


% However, it will take some time to prove Theorem~\ref{thm:unit}.